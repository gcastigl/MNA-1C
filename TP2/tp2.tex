% Preview source code

%% LyX 2.0.3 created this file.  For more info, see http://www.lyx.org/.
%% Do not edit unless you really know what you are doing.
\documentclass[english]{article}
\usepackage[T1]{fontenc}
\usepackage[latin9]{inputenc}
\usepackage{amsmath}

\makeatletter
%%%%%%%%%%%%%%%%%%%%%%%%%%%%%% Textclass specific LaTeX commands.
\newenvironment{lyxcode}
{\par\begin{list}{}{
\setlength{\rightmargin}{\leftmargin}
\setlength{\listparindent}{0pt}% needed for AMS classes
\raggedright
\setlength{\itemsep}{0pt}
\setlength{\parsep}{0pt}
\normalfont\ttfamily}%
 \item[]}
{\end{list}}

\makeatother

\usepackage{babel}
\begin{document}

\title{Procesamiento de imágenes I++}

\maketitle
Este ejercicio tiene como objetivo analizar espectralmente una imagen
y utilizar la transformada discreta de Fourier para realizar filtrados
espaciales. La Trasnformada Discreta de Fourier de una secuencia bidimensional
$x_{n,m}$ de $N\times N$ (imagen), es:

\[
X_{l,k}=\sum_{n=0}^{N-1}\sum_{m=0}^{N-1}x_{n,m}e^{-i\frac{2\pi}{N}(nl+mk)}
\]


Siendo la transformada inversa discreta:

\[
x_{n,m}=\frac{1}{N^{2}}\sum_{l=0}^{N-1}\sum_{k=0}^{N-1}X_{l,k}e^{i\frac{2\pi}{N}(nl+mk)}
\]



\section{Ejercicios}


\subsection{Implementar un programa que compute la TDF 2D.\protect \\
}

<Implementar solución del ejercicio aqui>\\



\subsection{El archivo \emph{saturno} contiene una matriz de $400\times400$
pixeles y corresponde a niveles de intensidad luminosa comprendidos
entre $0$ y $255$. }

Para visualizar esta imagen en escala de grises, es necesario establecer
un mapa de color de 255 niveles. Por ejemplo en MATLAB, se puede leer
y visualizar así 
\begin{lyxcode}
x=load(\textquoteright{}saturno\textquoteright{});~

colormap(gray(255));~

image(x\textquoteright{});


\end{lyxcode}
Visualizando la Figura 1 que muestra una imagen del planeta Saturno,
capturada por la misión Voyager.


\subsection{Computar la Transformada discreta de Fourier de la imagen original.
Armar las imágenes de $400\times400$ pixeles correspondientes a la
amplitud y la fase. }

Dichas imágenes deben verse como se muestra en la Figura 2 (Tener
en cuenta de mapear los valores de amplitud y fase al intervalo entero
$[0,255]$).


\subsection{Computar la Transformada inversa para reconstruir la imagen original
de $400\times400$ pixeles.}

<Poner solucion aqui>


\subsection{Considerar el efecto que producen los siguientes filtros $H_{k,l}$
de $400\times400$ pixeles en el dominio de las frecuencias (espaciales):}


\subsubsection{Filtro 1}

\[
H_{k,l}=\begin{cases}
0 & 0\leq k\leq400,\text{ }190\leq l\mathbf{\leq}210\\
1 & 0\leq l\leq400, 190\leq k\mathbf{\leq}210\\
1 & ,\text{ en todo otro caso}
\end{cases}
\]



\subsubsection{Filtro Gaussiano }

\[
H_{k,l}=exp(-0.1(k^{2}+l^{2}))
\]



\subsubsection{Filtro Dámero}

\[
H_{k,l}=\begin{cases}
0 & l+k\text{ es par}\\
1 & l+k\text{ es impar}
\end{cases}
\]

\end{document}

